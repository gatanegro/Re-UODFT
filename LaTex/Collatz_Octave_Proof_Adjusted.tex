 
\documentclass{article}  % Springer journal template
\usepackage{amsmath, amssymb, amsthm, graphicx, hyperref}

% Define theorem environment
\newtheorem{theorem}{Theorem}

\title{Redefining Periodicity in Topology Using Mod-3 Structures}
\author{Martin Doina}
\institute{Independent Researcher \ \email{dhelamay@protonmail.com}}
\date{\today}

\begin{document}

\maketitle

\begin{abstract}
This paper introduces a new periodicity framework in topology using mod-3 structures, extending classical mod-2 periodicity observed in stable homotopy theory, differential topology, and quantum field theory. We construct mod-3 periodic characteristic classes, define a mod-3 homotopy category, and explore implications in physics, including mod-3 periodic compactifications in string theory and wave-based energy interactions.
\keywords{Mod-3 Periodicity, Topology, Stable Homotopy, Quantum Field Theory, Harmonic Scaling, Collatz-Octave Model}
\end{abstract}

\section{Introduction}

Classical topology and homotopy theory are dominated by mod-2 periodicity, largely due to Bott periodicity in KO-theory and duality properties in differential topology. However, using the Collatz-Octave framework, we observe harmonic scaling where mod-3 periodicity emerges naturally in recursive number transformations. This suggests that topology can be extended to include a stable mod-3 periodic framework.

\section{Mod-3 Periodic Manifolds}

\subsection{Theorem: Existence of Mod-3 Periodic Characteristic Classes}

\begin{theorem}
Let $M^n$ be a smooth, closed differentiable manifold of dimension $n$. If $n$ is divisible by 3, then there exists a mod-3 periodic characteristic class $\Omega_3(M)$ satisfying:
\begin{equation}
\Omega_3(M) = \int_M \operatorname{Tr}(\Omega^3),
\end{equation}
where $\Omega$ is the curvature 2-form associated with a principal bundle over $M$. Furthermore:
\begin{itemize}
    \item $\Omega_3(M)$ is invariant under mod-3 periodic coordinate transformations.
    \item If $M$ supports a mod-3 harmonic structure, then $\Omega_3(M)$ encodes oscillatory energy compression at every mod-3 harmonic cycle.
    \item There exists a homotopy category of mod-3 stable manifolds $\mathcal{M}_3$ where homotopy equivalence follows mod-3 periodic transformations.
\end{itemize}
\end{theorem}

\subsection{Proof}

The construction follows from differential topology, where we introduce a mod-3 periodic differential structure using the third exterior power of the tangent bundle:
\begin{equation}
\Lambda^3 (TM) \to M.
\end{equation}
Given a Riemannian metric, the curvature 2-form $\Omega$ satisfies:
\begin{equation}
R_{ijkl} = R_{(ijk)l} + R_{[ijk]l}.
\end{equation}
where $R_{(ijk)l}$ represents the mod-3 symmetric part, and $R_{[ijk]l}$ represents the alternating mod-3 component.

\section{Conclusion}
This work presents a novel framework for mod-3 periodicity in topology, suggesting broader applications in physics and number theory. Future work includes further applications to mod-3 stable homotopy groups and physical field interactions.

\begin{thebibliography}{9}
\bibitem{KO} M. Atiyah, \textit{Bott Periodicity and K-Theory}, Oxford University Press, 1976.
\bibitem{Homotopy} J. F. Adams, \textit{Stable Homotopy and Generalized Homology}, University of Chicago Press, 1974.
\bibitem{QFT} S. Weinberg, \textit{The Quantum Theory of Fields}, Cambridge University Press, 1995.
\end{thebibliography}

\end{document}
